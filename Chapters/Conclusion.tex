% !Mode:: "TeX:UTF-8"

\chapter{全文总结与展望}\label{ch:7}

\section{论文工作总结}
本论文围绕“手眼协同的机械臂精准授粉技术”开展研究,针对当前农业场景中存在的花朵识别困难、位姿估计不准、机械臂顺柔控、末端精度不高等问题,提出了一套集视觉感知、位姿重建、柔顺控制与精度补偿于一体的精准授粉解决方案。系统以“感知—定位—控制—评估”闭环思想为指导,完成从花朵检测到末端执行器控制的全流程研究,具有较强的系统性与实用性。

本文的主要工作内容可以概括为下列四个方面:

(1)在视觉感知方面,本文设计了面向复杂农业环境的花朵目标分割方法,结合Mask-RCNN与YOLACT网络,在自建数据集上进行训练,提升了系统在强光、遮挡等条件下的分割精度与实时性。

(2)提出基于对称空间的位姿重建方法,通过深度去噪、三维坐标计算与手眼标定等步骤,实现了高精度的花朵空间定位,为后续柔顺控制提供了可靠基础。

(3)在控制策略设计方面,论文构建了一种两阶段伺服控制框架,结合三次样条插值生成关节空间轨迹,提升了机械臂在连续授粉任务中的运动柔顺性与安全性。

(4)考虑到末端精度对授粉效果的直接影响,本文引入基于Transformer的误差预测网络,通过学习末端的平移与旋转误差,进一步提升了授粉的效率和系统鲁棒性。
\section{后续工作展望}
尽管本研究已在精准授粉任务中取得了较为有效的成果,但仍存在一些值得进一步研究与改进的方向:

(1)复杂环境下的视觉感知鲁棒性仍需提升。当前模型在极端光照条件、严重遮挡情况下的分割与位姿估计精度仍有下降趋势,未来可考虑引入多模态感知技术(如RGB-IR或多光谱)与时序信息增强网络,提高系统在动态环境中的适应能力。

(2)控制系统的泛化能力有待加强。目前的控制策略主要针对番茄花朵设计,缺乏对其他作物品种的适应能力。未来研究可从模型迁移、任务参数化等方向出发,实现系统对多品种、多任务场景的通用控制。

(3)末端误差补偿机制尚未形成闭环反馈。当前Transformer网络基于前馈预测实现误差修正,未来可结合视觉伺服与力觉传感反馈,构建具备自适应与自校准能力的误差闭环补偿系统,进一步提高授粉精度。

(4)系统集成与移动作业能力仍有限。当前实验平台部署于静态环境中,后续可基于SLAM与路径规划算法,实现机械臂平台的自主移动授粉功能,朝着全自主、多目标的农业智能作业系统迈进。

综上所述,本文研究为实现农业自动化中的精准授粉提供了可行的技术路径与理论支持,具有良好的应用前景和研究推广价值。未来将在多模态感知、柔顺控制、自主决策与平台集成等方面持续深入探索,助力智能农业机器人技术的发展与落地应用。
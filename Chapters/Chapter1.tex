% !Mode:: "TeX:UTF-8"
%此为第一章节。
%[h]为hear代码所在位置,\caption为表注题注,\cref{}引用图表公式章节等,\cite为引用参考文献,\subfloat子图,\label标签,\begin{figure}图片环境,\begin{table}表格环境,\begin{equation}公式环境,\toprule三线表顶线,\cmidrule三线表中线,\bottomrule三线表底线,\begin{theorem}定理,\begin{proof}证明,\begin{corollary}推论,\begin{lemma}引理
    
\chapter{绪论}\label{ch:1}
\section{课题的研究背景与意义}
随着全球人口的不断增长和资源的日益紧张,农业领域面临着许多挑战。气候变化和劳动力短缺等问题,已经严重制约了农业生产的效率与可持续发展。为了解决这些问题,计算机技术和机器人技术的引入为农业生产带来了新的希望。计算机视觉在精准农业和自动化采摘方面的应用逐渐成熟。然而,在复杂场景下的农业自主授粉领域,精准授粉依然依然是一个难题。作为一个横跨了计算机、自动化和农业领域的自主授粉机器人的应用研究技术,它在现实场景中有非常丰富和实际的应用。自主授粉机器人的主要任务是在农业环境中,特别是在温室或室内作物生产中,自动执行作物授粉过程。这种机器人通过计算机视觉技术来识别和定位花朵的位置和朝向,然后使用专门设计的机械装置模仿自然授粉者,将花粉从一朵花转移到另一朵花。在整个过程中,机器人不仅能够自主识识别需要授粉的花朵以及它的位姿,并且能进行高效的精确操作,以确保授粉的效率和成功率。

自主授粉机器人在农作物的精准授粉中展现出巨大潜力。尤其在番茄这种经济作物的生产过程中,由于其花朵小而且朝向不规则,传统的方法是使用大面积的喷洒和大范围的伺服搜索授粉,造成了资源的浪费和效率低下,因此,开发一种能够在复杂环境下进行精准授粉的机器人系统显得尤为重要。在实际应用中,自主授粉机器人的视觉系统受到多种因素的影响,如目标物体的大小、颜色、朝向、光照以及空间分布等,这些因素均可能影响机器人的性能。此外,触碰控制也需考虑目标物体的位姿及是否有遮挡等因素,这些挑战需要通过高效的感知和智能的动态规划来克服,以确保机器人能够准确完成授粉任务。

本研究旨在通过先进的机器视觉和柔顺控制技术,解决机械臂在复杂环境下对番茄花进行精准授粉的技术难题。通过研究小目标检测、目标不明显的分割、背景复杂的目标检测与分割以及实时性坐标转换定位等关键技术,本课题不仅可以提高番茄授粉的自动化水平和效率,同时也有助于推动智能农业机器人的技术进步和应用普及。



\section{国内外研究现状}
\subsection{视觉感知}

在机器人自主操作任务中,视觉感知技术\cite{wade2013visual}是获取环境信息、实现目标定位与姿态估计的核心手段。根据感知系统获取空间信息的能力不同,现有视觉方法主要包括单目视觉\cite{celik2013monocular}、双目视觉\cite{blake2011binocular}与深度立体视觉\cite{howard2012perceiving}三类。

单目视觉方法依赖于单个RGB相机,具有结构简单、成本低廉等优点,适用于结构规则、工作平面明确的操作场景。Fang等人\cite{fang1billion}提出的GraspNet 提出了一种支持从单目 RGB 图像中估计物体抓取位姿的基准方法。该方法通过卷积神经网络提取图像特征,并在图像空间中密集采样候选抓取点,对每个点预测其对应的6D抓取姿态(位置与朝向)和抓取质量评分。该方法无需依赖深度图输入,仅利用RGB图像即可生成多个高质量的抓取候选,适用于复杂桌面场景中的通用物体抓取任务。Tekin等人\cite{tekin2018real}提出一种方法旨在解决工业场景中使用单目 RGB 图像进行实时物体 6D 姿态估计的问题,避免对深度图或迭代优化的依赖。其做法是构建一个端到端的单阶段卷积神经网络,直接从 RGB 图像中回归物体的三维位置和平移方向。为了提升精度,该方法不直接回归旋转参数,而是最小化物体关键点在图像中的投影误差,使网络通过2D-3D对齐来学习姿态信息,从而实现快速而准确的单目 6D 位姿估计,满足工业机器人对实时性与准确性的要求。

双目视觉通过设置两台位置已知的RGB相机,利用立体匹配算法计算目标在图像中的视差,从而估算三维空间位置。该方法能够有效弥补单目视觉无法获取深度的局限。Bontar和Kendall\cite{vzbontar2016stereo,kendall2017end}提出用卷积神经网络\cite{o2015introduction}学习左右图像块之间的相似性度量函数,以替代传统基于手工特征的方法,其做法是提取左图和右图的图像块特征,并通过网络输出它们的匹配相似度。Mur-Artal等人\cite{mur2017orb}提出了支持双目和RGB-D相机输入的一种通用、实时、精准的SLAM系统构建地图,通过跟踪-建图-闭环检测三个线程并行运行,完成摄像头位姿估计和稀疏地图重建。然而,双目视觉系统在光照不均、表面纹理缺乏或阴影区域时仍难以进行有效匹配,同时,其对图像匹配算法的精度和计算效率依赖较高,增加了系统复杂性。

深度立体视觉系统通过集成RGB相机与结构光等深度传感器,直接获取场景中各像素点的深度距离信息,在农业与服务机器人等领域中得到了广泛应用。Sa等人\cite{sa2017peduncle}提出了一种结合颜色特征与三维几何信息的甜椒果梗检测方法,旨在提升自动采摘机器人的识别精度与稳定性。该方法首先利用RGB图像提取颜色特征,再结合深度图重建果实的3D点云。

\subsection{花朵检测与实例分割} 
花朵的识别是授粉任务的第一步,其准确性直接影响后续定位与操作的成效。传统方法多依赖图像的颜色特征\cite{chen2008fast}与纹理特征\cite{ojala1999unsupervised}进行分割,结合形态学处理\cite{maragos1987tutorial}或边缘检测算法\cite{peli1982study}提取花朵区域。但此类方法对光照变化、背景干扰和目标遮挡敏感,鲁棒性有限,难以适应真实农业环境下的复杂条件。

近年来,随着卷积神经网络的发展,基于深度学习的目标检测与分割方法广泛应用于农业视觉任务。其中,Bargoti 等人\cite{bargoti2017deep}使用 Faster R-CNN\cite{girshick2015fast}实现了苹果园中果树花朵的自动检测,在大规模数据集上表现出很好的精度与召回率,标志着深度学习方法在农业视觉识别中的有效性。
He 等人提出的 Mask R-CNN\cite{he2017mask}通过在 Faster R-CNN 框架中引入掩膜预测分支,实现了像素级目标实例分割,在农业场景中被广泛用于花、果、叶等目标的精细识别,尤其适用于高精度需求的授粉与采摘任务。随后,Bolya 等人提出的 YOLACT\cite{bolya2019yolact} 模型采用掩膜系数与原型掩膜的并行结构,显著提高了推理速度,在保持一定精度的同时,实现了实时分割,适合高效率的授粉机器人系统。
\subsection{姿态估计}
实现目标物体高精度姿态估计具有很大挑战性。为提高精度,常使用额外传感器,如LiDAR或压力传感器,但会增加成本。一些方法基于6D物体位姿估计\cite{brachmann2014learning},构建模板扫描输入图像的不同位置,计算相似度得分,选取最佳匹配\cite{hinterstoisser2011gradient,cao2016real}。其他方法使用特征提取,CNN模型直接回归每个像素的三维坐标\cite{brachmann2014learning},或通过CNN描述特定物体姿态的后验概率密度\cite{krull2015learning},将观察图像与渲染图像进行比较。此外,有些方法在深度学习框架中结合模板和特征提取的优势,网络集成了自底向上的像素级标注和自顶向下的物体姿态回归\cite{xiang2017posecnn},以解耦方式预测物体的三维位置和旋转。这些方法通常使用深度传感器获取目标物体的深度信息。

然而,在无法获取深度信息的场景中,如农业授粉机器人,由于花朵的空洞区域,深度信息获取效果不佳。此外,训练时结合深度信息的模型无法直接回归预测目标物体的3D位置信息。上述基于模板和特征的方法在点云上训练,但在处理非凸多面体和非流形数据时存在挑战。现有的点云训练方法主要适用于规则形状,如凸多面体和流形。在处理农业环境中花朵等复杂不规则形状时,这些方法受到限制,难以直接应用传统的点云训练方法。此外,虚拟环境中的点云训练与现实应用之间存在差异,可能影响模型在实际场景中的性能。

\subsection{三维位姿感知与重建} 
授粉机器人需基于识别到的花朵位置与朝向规划末端执行器的运动轨迹,三维位姿感知因此成为关键环节。传统的双目视觉、结构光等方法虽可获取一定深度信息,但在户外或温室中易受光照干扰。RGB-D传感器(如RealSense、Kinect)因其成本低、结构简单,成为农业视觉感知的主流选择。

DenseFusion\cite{wang2019densefusion}通过融合RGB图像与深度图中提取的局部特征,实现六自由度姿态的高精度估计,在遮挡背景下表现优异。PVNet\cite{peng2019pvnet}引入关键点引导机制与RANSAC回归流程,即使目标部分遮挡也能完成稳定的姿态预测,提升了系统在动态环境下的鲁棒性。PoseCNN\cite{xiang2017posecnn}则将目标检测与姿态估计联合建模,通过多任务网络结构优化整体性能。
为了提升点云数据质量,Rodriguez 等人\cite{rodriguez2014clustering}提出一种基于密度聚类的点云去噪方法,有效滤除噪声点、填补空洞,提升了后续点云配准与建模的准确性。

\subsection{机器人技能学习模型}
机器人技能学习模型分为监督学习\cite{cunningham2008supervised}和强化学习\cite{wiering2012reinforcement}两类。监督学习以人为提供的技能示范为学习对象,分为离散模型和连续模型。离散模型根据单帧状态信息独立计算行为信息,结果相对简单,应用广泛。Mousavian等人\cite{mousavian20196}使用CNN模型直接从场景点云数据中预测机器人六自由度姿态,实现较好的抓取成功率。然而,离散模型可能导致时序动作不连续,引发问题。因此,连续模型成为研究热点。Yang等人\cite{yang2019learning}使用LSTM模型,将视觉信息序列转化为机器人动作序列。视觉Transformer也是处理时序图像信息的研究热点,Dasari等人\cite{dasari2021transformers}使用视觉Transformer让机器人从技能示范的视频中学习实际操作技能。此外,脉冲神经网络\cite{kabilan2021neuromorphic}(SNN)因其延时处理特性,也可用于机器人学习。Vijayan等人\cite{vijayan2022cerebellum}设计的类小脑机器人控制模型基于SNN,实现了系统运行的稳定性。


\section{本论文主要研究内容}
\begin{figure}[htb]
	\includegraphics[width=1 \textwidth]{total_content}
	\caption[论文主要研究内容]{论文主要研究内容} % 中括号中内容为插图索引中显示内容,可在题注内容过长时使用
	\label{fig:total_content}
\end{figure}
本论文围绕“手眼协同的机械臂精准授粉技术”开展系统研究,旨在解决温室环境中番茄等作物精准授粉过程中存在的花朵姿态不规则、定位不稳定、机械臂控制不柔顺、末端误差累积等关键技术问题。通过构建“感知—定位—控制—评估”一体化闭环系统,实现授粉任务的高效、稳定与智能化操作,主要研究内容如\cref{fig:total_content}所示。

针对农业环境中花朵目标尺寸小、形态变化大、背景复杂等视觉感知难题,论文设计并实现了一种结合Mask-RCNN与YOLACT的花朵实例分割方法。通过构建多样化的番茄花图像数据集,并在多种光照与遮挡条件下进行训练与评估,有效提升了花朵检测的精度与鲁棒性,为位姿估计与机械臂操作奠定了稳定的视觉基础。

论文提出了一种基于对称空间的三维位姿重建方法。该方法融合了深度图像去噪、三维点云坐标计算与高精度手眼标定技术,显著提升了目标花朵在三维空间中的位姿重建精度,尤其在深度信息缺失或不规则形状干扰下表现出良好的稳定性。同时,引入目标朝向估计模块,确保了末端执行器与花朵表面法线方向一致,从而提高了接触授粉的成功率。

在机械臂控制方面,论文设计了一种基于“粗到精”策略的两阶段视觉伺服控制方案,并结合三次样条插值算法生成关节空间内的柔顺轨迹,实现机械臂平稳、可控的授粉运动。该控制策略不仅有效规避了花朵破损风险,还兼顾了系统执行效率与动作安全性,确保机械臂在连续多目标操作中的稳定表现。

为解决实际授粉过程中末端授粉定位存在的误差问题,进一步提出了一种基于Transformer结构的误差预测网络。该网络利用视觉特征对机械臂末端在执行授粉任务中的平移与旋转偏差进行建模与误差补偿,从而提高末端操作的精准性。通过引入不同主干网络与位置编码方式的消融实验,验证了所提出网络结构在误差预测准确率与泛化能力方面的有效性。


\section{本论文的结构安排}
\cref{ch:1}:绪论,介绍了精准农业背景下机械臂自主授粉的研究意义,指出传统人工或粗放式授粉存在效率低下、资源浪费等问题,描述了实现复杂场景中高精度、自动化授粉的挑战。随后综述了国内外在视觉感知、花朵检测、姿态估计、位姿重建、技能学习等方面的研究进展,并明确指出本研究的创新点与技术路径,最后概述了论文的整体结构。

\cref{ch:2}:本章系统梳理了实现精准授粉所需的关键基础理论与算法技术,首先介绍物体姿态估计的数学模型(如齐次变换、欧拉角、四元数等)和三种典型估计方法(PoseNet、DenseFusion、PVNet),其次分析了机械臂运动规划中的传统方法(线性插值、Slerp、三次样条)与深度学习方法(DQN、DDQN、LSTM),为后续方法设计与实验提供了理论支持。

\cref{ch:3}:本章聚焦于复杂环境下的花朵检测与实例分割问题,采用Mask-RCNN和YOLACT两种主流深度分割网络对花朵区域进行像素级识别与提取。通过构建多样化的数据集,并在不同光照、遮挡条件下开展实验,评估模型的精度、速度及鲁棒性,为位姿估计与轨迹生成提供准确的视觉输入。

\cref{ch:4}:本章提出一种融合深度去噪、三维重建的位姿感知方法,重点解决花朵因结构不规则与深度缺失导致的姿态估计困难。实验中构建标定环境,采集多视角数据,通过对比去噪前后重建精度及手眼标定误差,验证该方法在提升花朵位姿识别精度方面的有效性。

\cref{ch:5}:本章围绕授粉过程的运动控制展开,提出一个“粗到精”的两阶段控制框架,结合三次样条插值生成末端轨迹,实现机械臂关节空间内的柔顺动作控制。构建授粉实验平台,开展多目标连续授粉实验,结果表明该控制策略在保证安全性的同时,有效提高了授粉成功率与操作稳定性。

\cref{ch:6}:本章针对机械臂末端因误差积累导致的授粉精度下降问题,引入基于Transformer的视觉误差预测网络,实现对平移与旋转偏差的建模与补偿。通过精心设计的数据集与训练策略,开展误差预测与消融实验,分析不同网络结构和位置编码对预测效果的影响,结果验证了方法在提升末端授粉精度方面的可行性。

\cref{ch:7}:本章总结了全文的研究内容与创新成果,肯定了系统在多目标精准授粉任务中的实用价值,并指出当前系统仍面临复杂光照适应性、模型泛化能力等问题。最后展望未来将在环境自适应性、多模态融合和智能控制策略等方面继续深入研究,以推动农业机器人精准作业技术的发展。



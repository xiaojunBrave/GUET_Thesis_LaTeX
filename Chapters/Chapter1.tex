% !Mode:: "TeX:UTF-8"
%此为第一章节。
%[h]为hear代码所在位置,\caption为表注题注,\cref{}引用图表公式章节等,\cite为引用参考文献,\subfloat子图,\label标签,\begin{figure}图片环境,\begin{table}表格环境,\begin{equation}公式环境,\toprule三线表顶线,\cmidrule三线表中线,\bottomrule三线表底线,\begin{theorem}定理,\begin{proof}证明,\begin{corollary}推论,\begin{lemma}引理
    
\chapter{绪论}\label{ch:1}


\section{课题的研究背景与意义}
随着全球人口的不断增长和资源的日益紧张,农业领域面临着许多挑战。气候变化和劳动力短缺等问题,已经严重制约了农业生产的效率与可持续发展。为了解决这些问题,计算机技术和机器人技术的引入为农业生产带来了新的希望。计算机视觉在精准农业和自动化采摘方面的应用逐渐成熟。然而,在复杂场景下的农业自主授粉领域,精准授粉依然依然是一个难题。作为一个横跨了计算机、自动化和农业领域的自主授粉机器人的应用研究技术,它在现实场景中有非常丰富和实际的应用。自主授粉机器人的主要任务是在农业环境中,特别是在温室或室内作物生产中,自动执行作物授粉过程。这种机器人通过计算机视觉技术来识别和定位花朵的位置和朝向,然后使用专门设计的机械装置模仿自然授粉者,将花粉从一朵花转移到另一朵花。在整个过程中,机器人不仅能够自主识识别需要授粉的花朵以及它的位姿,并且能进行高效的精确操作,以确保授粉的效率和成功率。

自主授粉机器人在农作物的精准授粉中展现出巨大潜力。尤其在番茄这种经济作物的生产过程中,由于其花朵小而且朝向不规则,传统的方法是使用大面积的喷洒和大范围的伺服搜索授粉,造成了资源的浪费和效率低下,因此,开发一种能够在复杂环境下进行精准授粉的机器人系统显得尤为重要。在实际应用中,自主授粉机器人的视觉系统受到多种因素的影响,如目标物体的大小、颜色、朝向、光照以及空间分布等,这些因素均可能影响机器人的性能。此外,触碰控制也需考虑目标物体的位姿及是否有遮挡等因素,这些挑战需要通过高效的感知和智能的动态规划来克服,以确保机器人能够准确完成授粉任务。

本研究旨在通过先进的机器视觉和柔顺控制技术,解决机械臂在复杂环境下对番茄花进行精准授粉的技术难题。通过研究小目标检测、目标不明显的分割、背景复杂的目标检测与分割以及实时性坐标转换定位等关键技术,本课题不仅可以提高番茄授粉的自动化水平和效率,同时也有助于推动智能农业机器人的技术进步和应用普及。



\section{国内外研究现状}
机器人环境感知技术主要依赖视觉手段,包括单目视觉、多目立体视觉和深度立体视觉。

\subsection{单目视觉}
使用单个RGB相机,由于无法直接获取距离信息,常用于平面操作任务。例如,Zhou等人(2018年)在桌面抓取任务中,利用CNN模型从单目图像预测抓取位置,通过设置相机正对水平桌面,获取目标的二维姿态,竖直距离通过基准工作平面确定。Hayat等人(2019年)基于几何分析方法,使用单目相机估计工业机器人关节姿态,实现状态监督。Lin等人(2010年)在移动机器人平台上,利用单目相机测量与障碍物的距离,实现避障功能。然而,单目视觉在三维空间操作任务中,因缺乏深度信息,存在精度限制。

\subsection{多目立体视觉}
通过两台相对位置已知的RGB相机,利用同一场景在两相机中的位置关系,计算图像点的空间位置。Hager等人(1995年)展示了使用双目视觉进行视觉伺服的机器人系统,表现出较高的精度和稳定性。Chiang等人(2011年)在三轴并联机器人系统中,使用双目视觉进行末端执行器和目标的跟踪,验证了双目视觉系统的有效性。然而,双目相机在阴影区域无法测得深度信息,且特征点匹配计算复杂度高,对算法性能依赖较大。

\subsection{深度立体视觉}
结合深度传感器和RGB相机,红外传感器直接获取目标点的距离信息。Guo等人(2020年)针对农业机器人采摘任务,使用Kinect深度相机的点云与模板点云匹配,完成目标姿态估计。但深度相机易受强反射、弱反射、小目标等因素影响,产生噪声和精度误差。为此,Xin等人(2017年)提出使用CNN对Kinect深度相机的深度信息误差进行校正的方法,提供了解决方案。

\subsection{机器人技能学习模型}
分为监督学习和强化学习两类。监督学习以人为提供的技能示范为学习对象,分为离散模型和连续模型。离散模型根据单帧状态信息独立计算行为信息,结果相对简单,应用广泛。Mousavian等人(2019年)使用CNN模型直接从场景点云数据中预测机器人六自由度姿态,实现较好的抓取成功率。然而,离散模型可能导致时序动作不连续,引发问题。因此,连续模型成为研究热点。Behrens等人(2021年)使用LSTM模型,将视觉信息序列转化为机器人动作序列。视觉Transformer也是处理时序图像信息的研究热点,Dasari等人(2020年)使用视觉Transformer让机器人从技能示范的视频中学习实际操作技能。此外,脉冲神经网络(SNN)因其延时处理特性,也可用于机器人学习。Zahra等人(2021年)设计的类小脑机器人控制模型基于SNN,实现了系统运行的稳定性。另一种连续模型是使用表征连续状态的模型输入,Groth等人(2021年)将连续的视觉图像处理得到动态图像,作为CNN模型输入预测行为信息,取得了较好的效果。

\subsection{强化学习模型}
通过机器人自主探索,根据任务完成情况给予奖励反馈,实现模型生成与优化。Deep Q-Network(DQN)是广泛使用的深度强化学习模型,可实现基于视觉图像的机器人离散控制。Kalashnikov等人(2018年)对DQN进行了扩展,通过交叉熵的应用实现了连续控制。此外,actor-critic深度强化学习方法也被用于机器人视觉操作任务的研究中。

\subsection{姿态估计}
实现机器人末端执行器的高精度姿态估计具有挑战性。为提高精度,常使用额外传感器,如LiDAR或压力传感器,但会增加成本。一些方法基于6D物体姿态估计,构建模板扫描输入图像的不同位置,计算相似度得分,选取最佳匹配。其他方法使用特征提取,CNN模型直接回归每个像素的3D坐标,或通过CNN描述特定物体姿态的后验概率密度,将观察图像与渲染图像进行比较。此外,有些方法在深度学习框架中结合模板和特征提取的优势,网络集成了自底向上的像素级标注和自顶向下的物体姿态回归,以解耦方式预测物体的3D位置和旋转。这些方法通常使用深度传感器获取目标物体的深度信息。然而,在无法获取深度信息的场景中,如农业授粉机器人,由于花朵的空洞区域,深度信息获取效果不佳。此外,训练时结合深度信息的模型无法直接回归预测目标物体的3D位置信息。上述基于模板和特征的方法在点云上训练,但在处理非凸多面体和非流形数据时存在挑战。现有的点云训练方法主要适用于规则形状,如凸多面体和流形。在处理农业环境中花朵等复杂不规则形状时,这些方法受到限制,难以直接应用传统的点云训练方法。此外,虚拟环境中的点云训练与现实应用之间存在差异,可能影响模型在实际场景中的性能。




\section{本论文的结构安排}
\cref{ch:1}:绪论。本章主要进行整体说明。

\cref{ch:2}:图片示例。

\cref{ch:3}:表格示例。

\cref{ch:4}:数学公式示例。

\cref{ch:5}:列表、算法、定理、证明插入示例。

\cref{ch:6}:全文总结与展望。本次研究工作进行总结,并根据全文研究过程中……。


